\section{Related Work}\label{sec:relw}
This paper is concerned with computing backbones of propositional formulae, which was oriented from coloring problem \cite{CJG2001}, with a wide range of practical applications such as MaxSAT \cite{MMBM2005}.

A number of backbones extraction algorithms have been proposed in recent years. All state-of-the-art backbones extraction approaches employ SAT solvers, MiniSAT\cite{MINISAT} for most of them.
Model enumeration\cite{MK2002,RSF2004} enumerates models of a given formula $\Phi$ one by one and updates the backbones estimation in each iteration. The negation of a model is added to $\Phi$ for blocking. It have to enumerate all models of $\Phi$ before the algorithm terminates, which is unnecessary costly.

Zhu et al, proposed an iterative SAT testing algorithm \cite{Z11}. The algorithm maintains an estimation of backbone. In each iteration, a clause that formed by the negation of backbone estimation is added to $\Phi$. If $\Phi$ is satisfiable, it means that at least one non-backbone literal is in the backbone estimation. Intersection between the model given by SAT solver and the backbone estimation indicates the non-backbone literal. In this way, non-backbone literal is removed, the process is repeated until $\Phi'$ is not satisfiable any longer. Along with the estimation, the clauses number of $\Phi$  is monotone increasing due to the continuously insertion in each iteration, which dramatically promote the complexity of $\Phi$. In other words, for each iteration, it takes longer CPU time than the last iteration.

The Core Based Algorithm presented in \cite{JLM15} is stable and effectiveness. It considers complementing of the model as assumptions input for SAT solver in each iteration. If $\Phi$ is unsatisfied under given assumptions, a core is returned by SAT solver to indicate reasons. According to the implementation of MiniSAT 2.2 \cite{MINISAT}, the reason is a part of the given assumptions. Whenever there is exactly one literal in the reason, the literal is a backbone. If there is more than one literal in the reason, they will be marked as visited. If every literal in an iteration is visited, iterative SAT testing will be invoked to test the rest of unmarked literals. According to the author, for the lower percentages of backbones, Core Based Algorithms are significantly better. When the percentage of backbone is over 25\%, Core Based Algorithm behave very similarly Iterative SAT Testing Algorithm.
