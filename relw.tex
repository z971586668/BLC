\section{Related Work}\label{sec:relw}
% This paper is concerned with computing backbones of propositional formulae, which was oriented from coloring problem \cite{CJG2001}, with a wide range of practical applications such as MaxSAT \cite{MMBM2005}.

A number of backbones extraction algorithms have been proposed in recent years. SAT solvers, especially MINISAT \cite{MINISAT}, employed in all state-of-the-art backbones extraction approaches.

Model enumeration\cite{MK2002,RSF2004} enumerates models of a given formula $\Phi$ one by one and updates the backbone estimation in each iteration. The negation of the model as a blocking clause is added to $\Phi$. It had to enumerate all models of $\Phi$ before the algorithm terminates, which is unnecessary costly.
Zhu et al, proposed an iterative SAT testing algorithm \cite{Z11} which is more efficient than model enumeration. At the first step, this algorithm assigned the first model returned from SAT solver to backbone estimation, which need less memory to compute backbone.
%The algorithm maintained an estimation of backbone. In each iteration, a clause that formed by the negation of backbone estimation is added to $\Phi$. If $\Phi$ was satisfiable, it implied that at least one non-backbone literal is in the backbone estimation. Intersection between the model given by SAT solver and the backbone estimation indicates the non-backbone literal. In this way, non-backbone literal is removed, the process is repeated until $\Phi'$ is not satisfiable any longer. Along with the estimation, the clauses number of $\Phi$  is monotone increasing due to the continuously insertion in each iteration, which dramatically promote the complexity of $\Phi$. In other words, for each iteration, it takes longer CPU time than the last iteration.

Janota et al, proposed an Iterative algorithm (one test per variable) \cite{JLM15}. For each iteration, it added only one unit clause to the original formula, which made the new formula easier. Inspired by this algorithm, our approach in this paper follows this idea.

The Core Based Algorithm presented in \cite{JLM15} is stable and effectiveness. The cb100 tool, as our main comparative object, was based on this algorithm. Instead of adding only one unit clause to the original formula, this algorithm added all the unit clause to the formula. It will dramatically accelerate SAT solving. Although there is a high possibility that the new formula is unsatisfiable, whenever there was only one literal in unsatisfiable reason of the new formula, this literal is a backbone literal. In this way, it was able to find backbone literals with little cost. In this paper, the author showed that for formulae with backbone percentages lower than 25\%, Core Based Algorithms are better.
%significantly better. When the percentage of backbone is over 25\%, Core Based Algorithm behave very similarly Iterative SAT Testing Algorithm.


Kilby et al proposed \cite{KST2005} that there was little overlap between backbone and backdoors.

In \cite{CZW2002}, they proposed an algorithm to discover backbone as well as fat variables. Variables that are absent from every optimal solution are fat variables. They proved the validity of the limit-crossing concept as well as other related properties. Then, they applied limit-crossing to Asymmetric Traveling Salesman Problem (ATSP). The experimental results demonstrated that improvements had been made compared with previous approach. 

In \cite{KKW2001}, they partitioned backbone into inadmissible and necessary group. For inadmissible group, literals were false in each satisfying variable assignment. For necessary group, literals were true in each satisfying variable assignment. They described and compared three algorithms for the search of the set of necessary and inadmissible variables, a basic iterative testing and two enhancements. The first one reused the result of satisfiability checks to get more models for free, the second one chose the decision variables for SAT solvers according to the previous information of inadmissible and necessary group.

In \cite{WS2001}, they concluded the correlation of backbone size with the problem of optimization and approximation, using graph coloring, Travel Salesman, Number partition and block words planning. And they suggested that it is necessary to eliminate some symmetries, perform trivial reductions and factor out the effective problem size.

In \cite{KSW2005}, they proved that it is intractable to approximate the backbone with any performance guarantee, assuming that P=NP and there is a limit on the number of edges falsely returned.
However, in practice, they could find much of the backbone using approximation, but, with a long runtime. Therefore, the suggestion was that use randomization and restarts in backbone searching and optimization searching.

In \cite{DD2001}, they proposed a heuristic search for backbone, and choose this backbone variables as branch nodes for the tree developed by a DPL-type procedure. Experiments showed that a significant performance improvement over the best current algorithms, and enhanced the scalability of the algorithm up to 700 variables.

