\section{Conclusions}\label{sec:conc}
This paper proposes a novel greedy-whitening based approach(\tool, for short) to compute backbone of a propositional formulae. 
\tool first computes an under-approximation of non-backbone, which can save SAT solving counts and running time of the formula.
Then, based on the under-approximation of non-backbone, \tool computes the over-approximation of backbone, which helps \tool to find a backbone literal earlier.
Finally, \tool iteratively tests literals to see if they are backbone literals, which is inspired by Iterative Algorithm. 

The experimental results show our approach is efficient, especially for industrial formulae which need longer time to compute the first model.
Future improvements to backbone computation algorithms include parallel approximations, automatically identification of partitions and more accurate community structure analysis.



\newpage
