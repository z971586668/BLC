\section{Conclusions}\label{sec:conc}
%This paper proposes a novel greedy-whitening based approach(\tool, for short) to compute backbone of a propositional formulae.
%\tool first computes an under-approximation of non-backbone, which can save SAT solving counts and running time of the formula.
%Then, based on the under-approximation of non-backbone, \tool computes the over-approximation of backbone, which helps \tool to find a backbone literal %earlier.
%Finally, \tool iteratively tests literals to see if they are backbone literals, which is inspired by Iterative Algorithm.

%The experimental results show our approach is efficient, especially for industrial formulae which need longer time to compute the first model.
%Future improvements to backbone computation algorithms include parallel approximations, automatically identification of partitions and more accurate community %structure analysis.


%\section{Conclusion}\label{sec:conc}
In this paper, we proposed a novel greedy-whitening based approach \tool to compute backbone of propositional formulae.
We implemented our approach in a tool \tool. Experimental results demonstrated that
\tool performs better than the state of the art tool \textit{cb100} on industrial formulae and hard random formulae.
From our experimental results, the time for creating first model can be used to measure hardness of Backbone computing for industrial formulae.
However, this observation does not hold for random formulae. Our experimental results demonstrated that community structures can be alternative.


