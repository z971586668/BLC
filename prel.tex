\section{Preliminaries}\label{sec:prel}

We fix a finite set  $\X$ of \emph{Boolean variables}.
A \emph{literal} $l$ is either a Boolean variable $x\in \X$ or its negation $\neg x$.
The negation of a literal $\neg x$ is $x$, i.e., $\neg\neg x=x$.
A \emph{clause} $\phi$ is a disjunction of literals $\bigvee_{i=1}^n l_i$, which may be regarded as
the set of literals $\{l_i\mid 1\leq i\leq n\}$.

A \emph{formula} $\Phi$ over $\X$ is a Boolean combination of variables $\X$.
We assume that formulae are given in conjunctive normal
form (CNF), namely each formula $\Phi$ is a conjunction of clauses $\bigwedge_{i=1}^n\phi_i$ which may be regarded as a set of clauses $\{\phi_i\mid 1\leq i\leq n\}$.
Given a formula $\Phi$, let $\var(\Phi)$ (resp. $\Lit(\Phi)$  and $\cls(\Phi)$) denote the set of variables (resp. literals and clauses) used in $\Phi$.
We use $\neg\Lit(\Phi)$ to denote the set $\{\neg l \mid l\in \Lit(\Phi)\}$.
The \emph{size} $|\Phi|$ of $\Phi$ is the number of literals of $\Phi$.
Given a formula $\Phi$ and a literal $l\in\Lit(\Phi)$,
let $\Phi_{l}\subseteq \Phi$ be the set of clauses $\{\phi\in\Phi\mid l\in\phi\}$.
%and $\Phi_\downarrow$ be the set $\{(l,\Phi_{\downarrow l})\mid l\in \Lit(\Phi)\}$.
%Given a clause $\phi$ of $\Phi$, let $\sharp(\phi,\Phi)$ be the count of occurrence of $\phi$ in the second components of tuples in $\Phi_\downarrow$.
%Given a set of clauses $\Phi_{l_x}$, let $\Phi_{l_x}^0 = \Phi_{\downarrow l}$, let $\Phi_{\downarrow l}^k$, $\Phi_{\downarrow l}^{k-1}\subseteq\Phi_{\downarrow l}^k\subseteq\Phi_{\downarrow 1}^{k+1}\subseteq\Phi (k \geq 1)$ be the set of clauses $\{\phi\in\Phi \mid \exists l\in\Lit(\Phi_{\downarrow l}^{k-1}), \neg l\in\Lit(\phi)\} (k \geq 1)$.

%A \emph{partial assignment} is a partial function $\lambda:\X\rightarrow \{0,1\}$ which assigns to each defined variable a Boolean value. Let $\lambda_x$ denote the partial assignment such that for every $y\in \X$, $\lambda_x(y)=\lambda(x)$ if $x=y$, otherwise $\lambda_x(y)$ is undefined.
%A \emph{complete assignment} is a partial assignment in which all the variables are defined.

An \emph{assignment} is a function $\lambda: \X \rightarrow \{0,1\}$, where $1$ (resp. $0$) denotes true (resp. false).
Given an assignment $\lambda$, let $\lambda[x\mapsto b]$ for some $b\in \{0,1\}$ be the assignment which is equal to $\lambda$
except for $\lambda[x\mapsto b](x)=b$.
An assignment $\lambda$ \emph{satisfies} a formula $\Phi$, denoted by $\lambda\models \Phi$, iff assigning $\lambda(x)$ to $x$ for $x\in\var(\Phi)$ makes $\Phi$ true. An assignment $\lambda$ is a \emph{model} of $\Phi$ if $\lambda\models \Phi$.
A formula $\Phi$ is \emph{satisfiable} iff there exists an assignment $\lambda$ such that $\lambda\models \Phi$.
%A variable $x\in\phi$ \emph{satisfies} a formula $\Phi$, denoted by $x\models\Phi$, iff assigning $1$ to $x$ makes with its value $\lambda(x)$.
Given a formula $\Phi$, the \emph{satisfiability problem} is to decide whether $\Phi$ is satisfiable or not.
%Whenever convenient, a formula $\Phi$ is seen as a set $\cls(\Phi)$ of clauses, a clause $\phi$ is seen as a set $\Lit(\phi)$ of literals, and an assignment %$\phi$ is seen as a set of clauses.
%In the rest of this section, we will introduce several technical concepts, amendable to our approach.

\smallskip
%Given an unsatisfiable Boolean formula $\Phi$, an \emph{unsatisfiable core} is a subset of clauses $C\subseteq \cls(\Phi)$ such that
%$\Phi_C$ is still unsatisfiable. An unsatisfiable core $C$ of $\Phi$ is called a \emph{minimal unsatisfiable core} if
%for every $C'\subset C$, $\Phi_{C'}$ is satisfiable. Intuitively, unsatisfiable core is a minimal unsatisfiable core it is unsatisfiable, but removing any of %its clauses makes it satisfiable.

\begin{definition}[Backbone]
\label{def:backbone}
Given a satisfiable formula $\Phi$, a literal $l$ is a \emph{backbone literal} of $\Phi$ iff for all assignments $\lambda$ such that $\lambda\models\Phi$,
$\lambda\models l$. The \emph{backbone} $\BL(\Phi)$ of $\Phi$ is the set of backbone literals of $\Phi$.
\end{definition}



%\begin{definition}[Nonbackbone]
%\label{def:backbone}
%Given a satisfiable formula $\Phi$, a variable $x\in\var(\Phi)$ is a \emph{nonbackbone variable} of $\Phi$ iff there exist two satisfying assignments
%$\lambda_1$ and $\lambda_2$ such that $\lambda_1(x)\neq \lambda_2(x)$.
%A clause $\phi\in \Phi$ is \emph{nonbackbone clause} iff $|\var(\phi)|\geq 2$ and there is a nonbackbone variable of $\Phi$ in $\phi$.
%For an unsatisfiable formula $\Phi$, the set of nonbackbone variables is $\var(\Phi)$.
%\end{definition}
It is known that the backbone $\BL(\Phi)$ for each formula $\Phi$ is unique and a backbone literal $l$ of $\Phi$ must be in
$\BL(\Phi)$ \cite{JLM15}.
The backbone of an unsatisfiable formula can be defined as an empty set. Therefore, in this work, we focus on satisfiable formulae.
We will use $\NBL(\Phi)$ to denote the set
$\Lit(\Phi)\setminus \BL(\Phi)$.

\begin{theorem}
\label{thm:co-NP}\cite{Jan10}
Given a satisfiable formula $\Phi$ and a literal $l$, deciding whether $l$ is a backbone literal is co-NP-complete.
\end{theorem}

% \begin{definition}[Model]
% Given a satisfied formula $\Phi$, a model $\lambda$ is a truth assignment such that $\lambda\models\Phi$.
% \end{definition}

\begin{definition}[Satisfied literal]
Given a model $\lambda$ of the formula $\Phi$ and a clause $\phi\in\Phi$, for each literal $l\in\phi$, $l$ is a \emph{satisfied literal}
of $\phi$ iff $\lambda\models l$. $l$ is a \emph{unique satisfied literal} of $\phi$ if there is no satisfied literal $l'$ of $\phi\setminus\{l\}$.
\end{definition}

%\begin{definition}[Unique satisfied literal]
%Given a satisfied formula $\Phi$, a model $\lambda$, a clause $\phi\in\Phi$, a satisfied literal $l$, $l$ is a unique satisfied literal iff %there
%exist exactly only one satisfied literal $l$ in $\phi$ with the assignment of $\lambda$.
%\end{definition}

%\begin{example}
Consider the formula $\Phi=\{\neg x_1 \vee \neg x_2, x_1, x_3 \vee x_4\}$,
$\var(\Phi)=\{x_1, x_2, x_3, x_4\}$, $\Lit(\Phi)=\{\neg x_1, x_1, \neg x_2, x_3, x_4\}$, $\Phi_{\neg x_2}=\{\neg x_1 \vee \neg x_2\}$ and $\BL(\Phi)=\{x_1, \neg x_2\}$.
%\end{example}



%\begin{remark}\label{rem:na}
%Given a formula $\Phi$,
%a na\"{\i}ve approach for computing $\BL(\Phi)$ is to verify whether a set $B$ is a backbone or not by calling SAT solver, for every $B\subseteq \var(\Phi)$. %However, this naive approach has to call $2^{|\var(\Phi)|}$ times SAT solver in the worst case \cite{JLM15}.
%\end{remark}
